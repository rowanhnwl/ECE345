\section{Question 2}
\subsection{(a)}
The form of the certificate is V vertices and E edges representing the common subgraph G.
The verification algorithm first affirms that the number of edges in G is greater than k.
This can be done in polynomial time with BFS or DFS.
Next, the verifcation algorithm iterates over each edge $(u, v) \in G$ and checks that $(u, v) \in G_1 \land (u, v) \in G_2$.
This can be accomplished by using DFS or BFS to search for the edge in $G_1$ and $G_2$.
The DFS or BFS take $\mathcal{O}(V + E)$ time and at most E edges need to be iterated over.
This makes the overall runtime of the verifcation algorithm to be $\mathcal{O}(E^2 + VE)$.
Therefore, we can verify the certificate in polynomial time.
\subsection{(b)}
To show that B is NP-hard, we can show that $A \leq_P B$. 
Let $\langle G, k \rangle$ be an instance of $\textsc{clique}$,
we can construct an instance of largest-common-subgraph through
the following steps in polynomial time. 

First, we know that by deleting edges from a clique of size k, 
we can construct an arbitrary graph beacuse a clique of size k maximizes
the number of edges. Second, We know that the number of
edges associated with a clique of size k is bounded by $\mathcal{O}(k^2)$
so deleting these edges runs in polynomial time. In this graph, $G_1$, 
We can delete edges in the clique to make an arbitrary subgraph of size k.

Creating a copy of the subgraph takes polynomial time since the graph can be 
traversed in  $\mathcal{O}(V + E)$ time with DFS or BFS. Then a series of edges and nodes
can be added to the copied subgraph which also takes polynomial time to create a secondary 
graph $G_2$. We need to show the following:
\[\langle G, k \rangle \in \textsc{clique} \Leftrightarrow \langle G1, G2, k \rangle \in \textsc{longest-common-subgraph}\]

\textbf{Proof} of $\Rightarrow$:
If a clique of size k exists in G, then a subgraph of size at least k is guaranteed to be 
in both subgraph G1 and G2 because the clique was modified to have an arbitrary subgraph of size k, 
and then it was copied for graph G2. 
\textbf{Proof} of $\Leftarrow$:
If G1 and G2 both have a common subgraph of size greater than k, then connecting k nodes in subgraph
yields a graph with a clique of size k.