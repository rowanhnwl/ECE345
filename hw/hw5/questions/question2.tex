\section{Question 2}
\subsection{(a)}
Given two graphs $G_1 = (V_1, E_1)$ and $G_2 = (V_2, E_2)$, the form of the certificate is 
n pairs of vertices $v_1 \in V_1$ and $v_2 \in V_2$ and m pairs of edges $e_1 \in E_1$ and $e_2 \in E_2$ representing the common subgraph and an integer k.
The verification algorithm first affirms that the number of edge pairs is greater than k.
Next, for each pair of edges $(u_1, v_1)$ and $(u_2, v_2)$, the verifcation algorithm checks that $(u_1, u_2)$ and $(v_1, v_2)$ are pairs in the set of vertices.
The worst case runtime for this algorithm means that for each edge pair, every vertex pair needs to be traversed making the time complexity $\mathcal{O}(VE)$
where V and E represent the maximum size between $G_1$ and $G_2$ of their vertices and edges respectivly.
Therefore, we can verify the certificate in polynomial time.
\subsection{(b)}
To show that B is NP-hard, we can show that $A \leq_P B$. 
Let $\langle G, k \rangle$ be an instance of $\textsc{clique}$,
we can construct an instance of largest-common-subgraph through
the following steps in polynomial time. 

First, we know that by deleting edges from a clique of size k, 
we can construct an arbitrary graph beacuse a clique of size k maximizes
the number of edges. Second, We know that the number of
edges associated with a clique of size k is bounded by $\mathcal{O}(k^2)$
so deleting these edges runs in polynomial time. In this graph, $G_1$, 
We can delete edges in the clique to make an arbitrary subgraph of size k.

Creating a copy of the subgraph takes polynomial time since the graph can be 
traversed in  $\mathcal{O}(V + E)$ time with DFS or BFS. Then a series of edges and nodes
can be added to the copied subgraph which also takes polynomial time to create a secondary 
graph $G_2$. We need to show the following:
\[\langle G, k \rangle \in \textsc{clique} \Leftrightarrow \langle G1, G2, k \rangle \in \textsc{longest-common-subgraph}\]

Proof of $\Rightarrow$:
If a clique of size k exists in G, then a subgraph of size at least k is guaranteed to be 
in both subgraph G1 and G2 because the clique was modified to have an arbitrary subgraph of size k, 
and then it was copied for graph G2. 
Proof of $\Leftarrow$:
If G1 and G2 both have a common subgraph of size greater than k, then connecting k nodes in subgraph
yields a graph with a clique of size k.