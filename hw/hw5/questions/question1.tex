\section{Question 1}

\textbf{Proof of} $\omega(T_{\text{approx}}) \le 2(1 - {1 \over l})\omega(T_{\text{opt}})$\textbf{:}

Firstly, let us consider a clockwise traversal of an optimal Steiner tree, $T_{\text{opt}}$, with the minimum number of leaves.
We will partition this traversal into sections denoted by $C_i$ for the $i$th section.
Let each section be the set of edges between two consecutive leaves in the order of traversal such that, for $l$ total leaves:

\begin{equation}
    C_i =
    \begin{cases}
        E \in (r_i \leadsto r_{i + 1}) & i \le l - 1 \\
        E \in (r_i \leadsto r_1)       & i = l
    \end{cases}
    \label{eqn:ci}
\end{equation}

Where $r_i$ refers to the $i$th visited leaf.
From Eqn. \ref{eqn:ci}, it is clear that this traversal forms a cycle through $T_{\text{opt}}$ since the last section, $C_l$, connects the final and initial visited leaves.
Finally, let $A$ denote the concatenation of all $C_i$ into one set.\vspace{3mm}

\textbf{Lemma 1:} The set $A$ contains exactly two instances of each edge in $T_{\text{opt}}$.\vspace{3mm}

\textbf{Proof of Lemma 1 by induction:}

\textbf{Basis:} Let $T_2$ be a Steiner tree with two nodes.
Since the two nodes must be connected by a single edge, they will each be leaves.
Because the clockwise traversal implies that there will be one section for each leaf, $T_2$ will produce two sections ($C_1$ and $C_2$).
Section $C_1$ will travel along the edge from the first node to the second, and $C_2$ will be the reverse.
Therefore, $C_1$ and $C_2$ each contain the same single edge with $A_{T_2} = \{C_1, C_2\}$ containing the edge exactly twice.
It has thus been shown that Lemma 1 holds for the basis. \\
\textbf{Hypothesis:} For a Steiner tree with $n$ nodes, $T_n$, there will be exactly two instances of each edge (in $T_n$) in $A_{T_n}$.\\
\textbf{Inductive Step:} Assuming the hypothesis is true, we must prove that this property will hold for $T_{n+1}$.
    To do this, we examine the different mechanisms in which a new node can be added to the Steiner tree $T_n$:
    \vspace{-2mm}
    \begin{enumerate}
        \item Adding a non-leaf node\vspace{-3mm}
        \item Adding a leaf node connected to a non-leaf node\vspace{-3mm}
        \item Adding a leaf node connected to leaf node\vspace{-3mm}
    \end{enumerate}

    A `leaf node' denotes a node that, when inserted, will only be connected to one other node in the tree.
    For \textbf{case 1}, we grow the tree by adding a new node that will not be a leaf.
    To preserve the properties of the tree, this must be done by bisecting an existing edge with the new node.
    Each instance of the edge in $A_{T_n}$ will simply be split into two different edges, allowing $A_{T_{n + 1}}$ to obey the hypothesis.
    Next, for \textbf{case 2}, the tree is grown by connecting a new leaf node to a non-leaf node.
    By increasing the total number of leaves by one, this will bisect some section $C_i$, splitting it into two new sections.
    Each of these new sections will have exactly one instance of the edge that was added to support the new leaf node.
    Therefore, in this case the only change to $A_{T_n}$ is adding exactly two instances of the new edge.
    It has thus been shown that case 2 respects the hypothesis.
    Finally, in \textbf{case 3} the tree grows from the addition of a new leaf node connected to an existing leaf node.
    By adding a new leaf in such a way, the existing leaf will have two adjacent nodes, thereby losing its leaf property.
    This addition then implies that the total number of leaves in the tree, $l$, will remain the same.
    Therefore, this addition simply extends two sections, $C_i$ and $C_{i+1}$, by the same edge, adding two instances of this new edge to $A_{T_n}$.
    Since all three cases preserve the hypothesis, $A_{T_{n + 1}}$ will have exactly two instances of each edge in $T_{n+1}$.

    \textbf{Q.E.D (Lemma 1)} \\

    Now that it has been shown that $A$ contains exactly two instances of each edge in $T_{\text{opt}}$, consider the largest section in $A$, denoted as $C_{\text{max}}$.
    Let us define the following set:
    \[P = A \text{\textbackslash} \{C_{\text{max}}\}\]

    In the \textbf{minimal} case, $\omega(C_{\text{max}}) = \omega(C_i), \forall i \in [1, l]$.
    By definition, this occurs when all $C_i$ have the same weight.
    In this case, since each section takes up an equal fraction of the weight of $A$, an upper bound can be derived for the weight of $P$:

    \[C_{\text{max}} \ge \frac{\omega{(A)}}{l}\]
    \[\omega{(P)} \le \omega{(A)} - \frac{\omega{(A)}}{l}\]
    \[\omega{(P)} \le (1 - \frac{1}{l})\omega{(A)}\]

    Finally, by Lemma 1, the weight of $A$ should be exactly double the weight of $T_{\text{opt}}$.
    This leads to an upper bound on the weight of $P$ with respect to the weight of $T_{\text{opt}}$:

    \begin{equation}
        \omega{(P)} \le 2(1 - \frac{1}{l})\omega{(T_{\text{opt}})}
        \label{eqn:p}
    \end{equation}

    Let us now examine the Steiner tree approximation algorithm.\vspace{3mm}

    \textbf{Lemma 2:} $\omega{(T_{\text{approx}})} \le \omega{(G_2)}$\vspace{3mm}

    \textbf{Proof of Lemma 2:}
    Since $G_2$ is a minimum spanning tree of $G_1$, a complete graph, its total weight must be less than $G_1$.
    Next, the weight of $G_2$ after being mapped back into $G$ remains the same since the edges of $G_2$ will be replaced by equally-weighted paths.
    With $G_2$ being mapped into $G$, the minimum spanning tree $G_4$ of $G_3$ will have a weight less than or equal to $G_2$.
    Finally, the weight of $G_4$ may be further reduced in $T_{\text{approx}}$ by removing any leaves that are not vertices in $R$.
    Therefore $\omega{(T_{\text{approx}})} \le \omega{(G_2)}$.

    \textbf{Q.E.D (Lemma 2)}\\

    The total path weight between any two required nodes (in $R$) in $P$ is \textbf{at minimum} the smallest path weight between them in $G$.
    Equivalently, the lightest path in $G$ between two nodes in $R$ has the weight of their connecting edge in $G_1$.
    Therefore, consider removing any connections in $G_1$ whose equivalent connection in $P$ requires passing through a separate node in $R$.
    The result is a graph ${G_1}^\prime$ that spans $G_1$ and has a weight (by Inequality \ref{eqn:p}): 
    \[\omega{({G_1}^\prime)} \le \omega{(P)} \le 2(1 - \frac{1}{l})\omega{(T_{\text{opt}})}\]
    This is because, by removing the aforementioned connections in $G_1$, we now guarantee that any path from one required node to the next is as light as possible.
    Finally, since $G_2$ is a minimum spanning tree of $G_1$, $\omega{(G_2)} \le \omega{({G_1}^\prime)}$.
    Therefore, incorporating Lemma 2 and Inequality \ref{eqn:p}, we are left with the following relationship:
    \[\omega{(T_{\text{approx}})} \le \omega{(G_2)} \le \omega{({G_1}^\prime)} \le \omega{(P)} \le 2(1 - \frac{1}{l})\omega{(T_{\text{opt}})}\]
    Simplifying, we achieve the bound:
    \[\omega{(T_{\text{approx}})} \le 2(1 - \frac{1}{l})\omega{(T_{\text{opt}})}\]\\

    \textbf{Q.E.D}