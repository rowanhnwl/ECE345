\section{Question 4}
    Note: Leftist heap will be abbreviated as LH.
    \subsection{(a)}
    Let $s$ represent the largest complete sub-tree of an LH $L$ starting from the root. Since the rank of the root of $L$ will be the length of the shortest path from the root to the leaf, the height of $s$ will have a height of the rank of the root of $L$ (otherwise $s$ would not be complete). If $m$ is the number of nodes in $s$, the height of $s$ will be $\mathcal{O}(\lg{m})$ which will be the same as the rank of the root. If $n$ is the number of nodes in $L$, then since $s$ is a sub-tree of $L$, $n \geq m \implies \lg{n} \geq \lg{m} \implies$ the rank of the root of an LH is $\mathcal{O}(\log{n})$. QED
    
    \subsection{(b)}
    From (a), we know that the rank of the root of an LH is $\mathcal{O}(\log{n})$ which is the same as the length of the rightmost path. We also know that to merge two sorted sequences using \textbf{MERGE} (CLRS, 4th, page 38), it takes $\Theta(n)$. If the size of two leftist heaps $l_1$ and $l_2$ have sizes $n_1$ and $n_2$, then to merge the rightmost paths of $l_1$ and $l_2$, the \textbf{MERGE} procedure will have to iterate over $\mathcal{O}(\log{n_1}) + \mathcal{O}(\log{n_2}) = \mathcal{O}(\log{n})$ elements. Therefore, to merge $l_1$ and $l_2$, it takes $\mathcal{O}(\log{n})$ time. To show that the order invariant is maintained, suppose that the root of an LH $l_1$ with no right child is being added to the right child of the root of another LH $l_2$ with its right child removed in the LH merge procedure (where merging two LHs splits both of them into sub-trees with their root's right child removed). Since the key of the root of $l_1$ is larger (because we are assuming this is happening in the merge procedure) and all other nodes of $l_1$ are larger than its root by the definition of an LH, all other nodes in $l_1$ will be larger than the root of $l_2$. The left side of $l_2$ and $l_1$ are not changed and they were already  LHs, so the order invariant is already maintained for the left side of both trees.
    QED
    
    \subsection{(c)}
     Since merging the rightmost paths of two trees $t_1$ and $t_2$ involves recursively removing the right child from the produced sub-trees, merging the rightmost paths of two trees can be modelled as merging the right path of m sub-trees with their root not having a right child. After splitting $t_1$ and $t_2$ into m sub-trees, only the right child of the root of each sub-tree will be changed. Take one of the subtrees $m_1$. If a new tree of arbitrary shape and size is set as the right child of $m_1$, then the left child subtree of $m_1$ would not have been changed. Since the rank of a node is defined as $1 + min{rank(left(x)), rank(right(x))}$ which depends on the rank of the children and the left subtree of $m_1$ was not changed, the rank of the left child does not change as well. In contrast, the rank of the right child of $m_1$ will be of arbitrary size as it will have its right child assigned to an arbitrary sized tree in the recursive call to merge. QED

    \subsection{(d)}
    From (b) we know that the process of merging the rightmost path of two LHs takes $\mathcal{O}(\log{n})$ time and that it keeps the order invariant. Therefore, we need to show that the rank update step takes $\mathcal{O}(\log{n})$ time and that it maintains the balance invariant and the order invariant.

    Suppose we are left with an LH l of size n that is the result of merging the rightmost path of two LHs.
    We will prove that the sub-tree $T_k$ of size $k$ is an LH after applying the rank update step using induction.
    
    \textbf{Inudction Hypothesis:} In the merge procedure, by applying the rank update step to the root node of a tree $T_k$ of size $k$, $T_k$ is a LH for $k > 0$.
    \textbf{Basis:} For k = 1, the tree has one root node with no children and it trivially maintains the balance and order invariant, so it is a LH.
    \textbf{Inductive Step:} $T_{k+1}$, can have zero, one, or two children that are sub-trees, so the number of nodes in these sub-trees will be less than the number of nodes in $T_{k+1}$. By the IH, since the number of nodes is less than k + 1, then these sub-trees are LHs. From (b) we have already shown that the merge procedure ensures that the order invariant is maintained. The merge procedure will move whichever child has a smaller rank to the right child so that the balance invariant is maintained and it does not change the height of the children, so the order invariant from (b) is not modified. Therefore, the resulting tree $T_{k+1}$ is also a LH.

    From (a), we also know that the length of the rightmost path of the merged LH will be $\mathcal{O}(\log{n_1}) + \mathcal{O}(\log{n_2}) = \mathcal{O}(\log{n})$. From (c) we know that merging the rightmost paths will only change the rank of nodes on the rightmost path, so the rank update step only needs to be applied to the rightmost path. If the rightmost path of the merged tree is traversed from the bottom rightmost leafnode, then $\log{n}$ nodes will be traversed with their children being optionally swapped. Since swapping the child of each parent is just swapping two pointers which takes $\Theta(1)$ time, the rank update step takes $\mathcal{O}(\log{n})$ time.
    QED

    \subsection{(e)}
    To implement \textbf{DeleteMin} and \textbf{Insert}, we can utilize the \textbf{Merge} procedure that runs in $\mathcal{O}(\log{n})$ time. The subtree starting at each node of an LH is also an LH or else the order and balance invariant would not be maintained which is why we can use the merge procedure here. Both \textbf{DeleteMin} and \textbf{Insert} run in $\mathcal{O}(\log{n})$ time because they call the \textbf{Merge} procedure once. For DeleteMin, since we know that the root node is the smallest node in the tree by the order invariant, we can remove the root node, so we are left with the root's two child sub-trees that are also LHs which can have the merge procedure applied to them to create a single LH. For Insert, we can imagine that the node being inserted is itself a LH. Since it needs to be placed into another LH, we can also use the Merge procedure to create a final LH.
    The following pseudocode assumes that the merge procedure returns the merged LH and that each node of an LH is also an LH
        \begin{codebox}
        \Procname{$\proc{DeleteMin}(H)$}
        \li $\id{l} \gets root.left$
        \li $\id{r} \gets root.right$
        \li $H \gets \proc{Merge}(l, r)$
        \end{codebox}
        \begin{codebox}
        \Procname{$\proc{Insert}(H, i)$}
        \li $H \gets \proc{Merge}(H, i)$
        \end{codebox}
        