
\section{Question 3}
    \subsection{(a)}
    \begin{spacing}{1.2}
        Assume $S_n$ represents sequence $S$ with size of $n$ elements:
        \begin{codebox}
            \Procname{$\proc{FindIntersection}(A_n, B_n, n)$}
            \li $I =$ empty list
            \li $a = 0$, $b = 0$, $most\_recent = -1$
            \li \While $a < n$ and $b<n$
                \Do
                    \li \If $A[a]$ is $B[b]$
                    \Then
                        \li \If $A[a]$ is not $most\_recent$
                        \Then
                            \li $I.insert(A[a])$
                            \li $most\_recent = A[a]$
                        \End
                        \li $a = a + 1$
                        \li $b = b + 1$
                    \li \ElseIf $A[a] < B[b]$
                    \Then
                        \li $a = a + 1$
                    \li \ElseIf $A[a] > B[b]$
                    \Then
                        \li $b = b + 1$
                    \End 
                \End
            \li \Return $I$
        \end{codebox}
    \end{spacing}
    \vspace{5mm}
    \textbf{Description and Complexity:}
    
    In this algorithm, we use pointers in both of our arrays to find the common elements between them.
    The algorithm is a \textbf{while} loop to find the common elements until one of the pointers gets to the end of the array (lines 3 - 13).
    This is simply done by inserting the number whenever the elements that our pointers point at have the same value.
    And whenever the values are different, the pointer pointing at the smaller number would be incremented, hoping that it 
    would point at a number closer (or maybe equal) to the other one. It is also necessary to keep updating the most recent value
    that was added to the list to avoid adding duplicates. In terms of space complexity, we are adding a new array 
    for the intersection with a maximum size of $n$ (if the arrays are identical), resulting $\mathcal{O}(n)$ complexity. In terms of time, both pointers
    increment towards the end of their arrays through the loops, which means it has $\mathcal{O}(n+n)=\mathcal{O}(n)$ time complexity in the worst-case scenario. \\
    
    \textbf{Inductive Proof of Algorithm Correctness:} \\
    \textbf{Basis:} \\
    Consider the following arrays: $A=\{1\}$, $B=\{1\}$. In this case, our first value would be inserted during the first loop. Since both of our pointers are incremented up to
    the size of the arrays, the \textbf{while} loop is ended. This means we are left with $I=\{1\}$ which is the correct answer. \\
    \textbf{Inductive Hypothesis:} \\
    Assume that the algorithm is correct for the two sorted arrays $A=\{a_0, a_1, \ldots, a_n\}$ and $B=\{b_0, b_1, \ldots, b_n\}$ when $1 \leq n \leq k-1$.
    Now we need to prove that it is still correct for $A={a_0, a_1, \ldots, a_k}, B={b_0, b_1, \ldots, b_k}$. Let's also assume that the pointers
    of arrays $A$ and $B$ are represented by $a$ and $b$ respectively. \\ 
    \textbf{Inductive Step:} \\
    Based on our inductive hypothesis, we can assume that the algorithm works correctly up to either $a_{k-1}$ or $b_{k-1}$. Let's assume that the pointer of 
    array $A$ reaches the end first, which means that $\max(A) \leq \max(B)$. This is because the pointer of an array is incremented if and only if the value of its number
    is less or equal to the value of the other pointer. In other words, pointer of $A$ can only get from $k$ to $k+1$ when the last element of $A$ is less or equal to
    the current element of $B$ ($A[a_k] \leq B[b]$). Here we have two possibilities to analyze: \\
    1. If $A[a_k] = B[b]$, then the values would be compared and added to the intersection list before the end of the \textbf{while} loop. And since this is the last element of
    array $A$, there wouldn't be any common element left. \\
    2. If $A[a_k] < B[b]$, then all the common elements have already been inserted since any number after $B[b]$ would be greater or equal to $B[b]$.\\
    The same scenario would be valid in the opposite case, when the pointer of $B$ reaches the end first instead, which means that the algorithm is correct in either case.\\
    \textbf{Q.E.D} \\

    \textbf{Proof of Termination:} \\
    The \textbf{while} loop ends when either of the pointers reaches the end of array (i.e. the pointer would be equal to the size of array, $n$). In each loop, at least one of the pointers is
    getting incremented, meaning that it will take less than $n+n$ loops for the termination of the \textbf{while} loop.
    \pagebreak
    \vspace{-7mm}
    \subsection{(b)}
    \vspace{-2mm}
    \begin{spacing}{1.2}
        Assume $S_n$ represents sequence $S$ with size of $n$ elements:
        \begin{codebox}
            \Procname{$\proc{FindUnion}(A_n, B_n, n)$}
            \li $U =$ empty list
            \li $a = 0$, $b = 0$, $most\_recent = -1$
            \li \While $a < n$ and $b<n$
                \Do
                    \li \If $A[a]$ is $B[b]$
                    \Then
                        \li \If $A[a] > most\_recent$
                        \Then
                            \li $U.insert(A[a])$
                            \li $most\_recent = A[a]$
                        \End
                        \li $a = a + 1$
                        \li $b = b + 1$
                    \li \ElseIf $A[a] < B[b]$
                    \Then
                        \li \If $A[a] > most\_recent$
                        \Then
                            \li $U.insert(A[a])$
                            \li $most\_recent = A[a]$
                        \End
                        \li $a = a + 1$
                    \li \ElseIf $A[a] > B[b]$
                    \Then
                    \li \If $B[b] > most\_recent$
                        \Then
                            \li $U.insert(B[b])$
                            \li $most\_recent = B[b]$
                        \End
                        \li $b = b + 1$
                    \End 
                \End
            \li \If $a$ is $n$
            \Then
                \li \For $b$ to $n-1$
                \Then
                    \li \If $A[n-1]<B[b]$ and $B[b]>most\_recent$
                    \Then
                        \li $U.insert(B[b])$
                        \li $most\_recent = B[b]$
                    \End
                \End
            \li \ElseIf $b$ is $n$
            \Then
                \li \For $a$ to $n-1$
                \Then
                    \li \If $A[a]>B[n-1]$ and $A[a]>most\_recent$
                    \Then
                        \li $U.insert(A[a])$
                        \li $most\_recent = A[a]$
                    \End
                \End
            \End
            \li \Return $U$
        \end{codebox}
    \end{spacing}
    %\vspace{12mm}
    \textbf{Description and Complexity:}
    
    In this algorithm, we use pointers in both of our arrays to find the union of them.
    The algorithm starts with a \textbf{while} loop to insert all unique elements until one of the pointers gets to the end of the array (lines 3 - 19).
    This is simply done by inserting the number whenever the elements that our pointers point at have different values.
    And whenever the values are the same, only one of them would be added while both pointers get incremented. After the \textbf{while} loop,
    as one of the pointers get to the end, the algorithm starts a \textbf{for} loop in the other array to insert all the values left (lines 20 - 29).
    It is also necessary to keep updating the most recent value that was added to the list to avoid adding duplicates. In terms of space complexity, we are adding a new array 
    for the union with maximum size of $n+n$ (if the arrays have no common elements), resulting $\mathcal{O}(n)$ complexity. In terms of time, both pointers
    increment towards the end of their arrays through the loops, which means it has $\mathcal{O}(n+n)=\mathcal{O}(n)$ time complexity. \\
    
    \textbf{Inductive Proof of Algorithm Correctness:} \\
    \textbf{Basis:} \\
    Consider the following arrays: $A=\{1\}$, $B=\{2\}$. In this case, the first value that would be added is $1$, since it is less than $2$. After that, as the
    pointer in array $A$ reaches the end, we move to the \textbf{for} loop for array $B$, inserting all unique values from the current pointer to
    the end of this array (in this case, only one element). This means we are left with $U=\{1, 2\}$ which is the correct answer. \\
    \textbf{Inductive Hypothesis:} \\
    Assume that the algorithm is correct for the two sorted arrays $A=\{a_0, a_1, \ldots, a_n\}$ and $B=\{b_0, b_1, \ldots, b_n\}$ when $1 \leq n \leq k-1$.
    Now we need to prove that it is still correct for $A={a_0, a_1, \ldots, a_k}, B={b_0, b_1, \ldots, b_k}$. Let's also assume that the pointers
    of arrays $A$ and $B$ are represented by $a$ and $b$ respectively. \\ 
    \textbf{Inductive Step:} \\
    Based on our inductive hypothesis, we can assume that the algorithm works correctly up to either $a_{k-1}$ or $b_{k-1}$. Let's assume that the pointer of 
    array $A$ reaches the end first, which means that $\max(A) \leq \max(B)$. This is because the pointer of an array is incremented if and only if the value of its number
    is less or equal to the value of the other pointer. In other words, pointer of $A$ can only get from $k$ to $k+1$ when the last element of $A$ is less or equal to
    the current element of $B$ ($A[a_k] \leq B[b]$). Here we have two possibilities to analyze: \\
    1. If $A[a_k] = B[b]$, then the values would be compared and added to the intersection list before the end of the \textbf{while} loop. And since this is the last element of
    array $A$, there wouldn't be any element in this array to be added to our list. The \textbf{for} loop would then add all the values in array $B$ while making sure there isn't any duplicate
    from $b+1$ up to $B[b_k]$ as expected. \\
    2. If $A[a_k] < B[b]$, then the value of $A[a_k]$ would be inserted to the list at the end of the \textbf{while} loop. And since this is the last element of
    array $A$, there wouldn't be any element in this array to be added to our list. The \textbf{for} loop would then add all the values in array $B$ while making sure there isn't any duplicate
    from $b$ up to $B[b_k]$ as expected. \\
    The same scenario would be valid in the opposite case, when the pointer of $B$ reaches the end first instead, which means that the algorithm is correct in either case.\\
    \textbf{Q.E.D} \\

    \textbf{Proof of Termination:} \\
    The \textbf{while} loop ends when either of the pointers reaches the end of array (i.e. the pointer would be equal to the size of array, $n$). In each loop, at least one of the pointers is
    getting incremented, meaning that it will take less than $n+n$ loops for the termination of the \textbf{while} loop. The \textbf{for} loop will also be terminated in maximum of $n$ loops as it gets
    incremented every single time with exit point of when it is equal to the size of the array.

    \subsection{(c)}
    \begin{spacing}{1.2}
        Assume $S_n$ represents sequence $S$ with size of $n$ elements:
        \begin{codebox}
            \Procname{$\proc{FindDiff}(A_n, B_n, n)$}
            \li $D =$ empty list
            \li $I = \proc{FindIntersection}(A_n, B_n, n)$
            \li $a = 0$, $i = 0$, $size=I.size$, $most\_recent = -1$
            \li \While $a < n$
                \Do
                    \li \If $i$ is $size$ or $A[a]<I[i]$
                    \Then
                        \li \If $A[a]$ is not $most\_recent$
                        \Then
                            \li $D.insert(A[a])$
                            \li $most\_recent = A[a]$
                        \End
                        \li $a = a + 1$
                    \li \ElseIf $A[a]>I[i]$
                    \Then
                        \li $i = i + 1$
                    \li \ElseNoIf
                        \li $a = a + 1$
                    \End 
                \End
            \li \Return $D$
        \end{codebox}
    \end{spacing}
    \vspace{5mm}
    \textbf{Description and Complexity:}
    
    In this algorithm, we first find the intersection between the two arrays using the algorithm provided in part a). Then, we use pointers in both of our 
    arrays ($A$ and $I$, the intersection) to find the unique elements that are not in array $B$.
    The algorithm includes a \textbf{while} loop to find the elements in array $A$ that are not in $I$ (i.e. not common with $B$) until the end of array $A$ (lines 4 - 13).
    This is simply done by inserting the number whenever the element pointed in array $A$ is smaller than the one pointed in $I$.
    And whenever its the other way around, the intersection array pointer would be incremented, hoping that it 
    would point at a number closer (or maybe equal) to the other one so we can decide whether it is just in $A$ or in both $A$ and $I$. It is also necessary to keep updating the most recent value
    that was added to the list to avoid adding duplicates. In terms of space complexity, we are adding a new array 
    for the difference with maximum size of $n$ (if the arrays have no element in common) and an array for finding the intersection, resulting $\mathcal{O}(n+n)=\mathcal{O}(n)$ complexity. In terms of time, both pointers
    increment towards the end of their arrays through the loops, which means it has $\mathcal{O}(n)+\mathcal{O}(n)=\mathcal{O}(n)$ time complexity. \\
    
    \textbf{Inductive Proof of Algorithm Correctness:} \\
    \textbf{Basis:} \\
    Consider the following arrays: $A=\{1\}$, $B=\{2\}$. In this case, there is no common element between the two, which means $I=\{ \O \}$.
    In the \textbf{while} loop, since the pointer is the same as the size (both equal to 0), the first element in $A$ is inserted and the pointer is incremented. 
    The algorithm ends here as the pointer of $A$ has reached the end of the array. This means we are left with $D=\{1\}$ which is the correct answer. \\
    \textbf{Inductive Hypothesis:} \\
    Assume that the algorithm is correct for the two sorted arrays $A=\{a_0, a_1, \ldots, a_n\}$, $B=\{b_0, b_1, \ldots, b_n\}$ and $I=\{i_0, i_1, \ldots, b_m\}$ when $1 \leq m \leq n \leq k-1$.
    Now we need to prove that it is still correct for $A={a_0, a_1, \ldots, a_k}, B={b_0, b_1, \ldots, b_k}$. Let's also assume that the pointers
    of arrays $A$ and $I$ are represented by $a$ and $i$ respectively. \\ 
    \textbf{Inductive Step:} \\
    Based on our inductive hypothesis, we can assume that the algorithm works correctly up to $a_{k-1}$. We know that $\max(A) \geq \max(I)$. This is because the elements of $I$ are already in $A$, meaning that the maximum value of $A$ cannot be smaller than
    the maximum value of $I$. Here we have two possibilities to analyze: \\
    1. If $A[a_k] = \max(I)$, then both pointers will be incremented until they both point at their last element. In that loop, since they values are equal,
    no element will be inserted and the \textbf{while} loop will be ended as the pointer $a$ would reach the end of the array \\
    2. If $A[a_k] > \max(I)$, then because of the \textbf{else if} statement, the pointer $i$ would be incremented until it reaches the end of the array $I$. Then, since $i$ is now equal to $size$, $A[a_k]$ would be
    inserted to the list as an element that is not commone between $A$ and $B$. \\
    Both cases will give us a correct result, which means that the algorithm is correct.
    \textbf{Q.E.D} \\

    \textbf{Proof of Termination:} \\
    The \textbf{while} loop ends when pointer of array $A$ reaches the end of the array (i.e. the pointer would be equal to the size of array, $n$). Until $I$ pointer reaches the end of the array, at least one of the pointers is
    getting incremented in our while loop. After that, only the $A$ pointer gets incremented, meaning that it will take less than $n+n$ loops for the termination of the \textbf{while} loop.