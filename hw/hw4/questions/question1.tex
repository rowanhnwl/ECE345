\section{Question 1}
	\subsection{(a)}
	To show that G and H share the same set of minimum spanning trees, we can show that for any sub-MST, they both share the same safe edges.
	In theorem 23.1 in CLRS 3rd edition, it is shown that the light edge crossing the cut of any cut of G that respects A is safe for A.
	The theorem uses the fact that since $(u,v)$ is a light edge, $w(u,v) \leq w(x,y)$ for another edge $w(x,y)$ crossing the cut. 
	This means that swapping $w(u,v)$ for $w(x,y)$ leads to $w(T') = w(T) - w(x,y) + w(u,v) \leq w(T)$.
	If both $w(x,y)$ and $w(u,v)$ get incremented by one when translating G to H, the above theorem shows that $(u, v)$ is still the safe edge for G and H
	since the inequality remains unchanged. Therefore, G and H share the same minimum spanning trees.

	\subsection{(b)}
		G and H do not share the same shortest paths between all pairs of vertices.
		\linebreak
		As a counter example, consider a graph G with a destination vertex $v_d$ and source vertex $v_s$.
		We need only to show that the shortest path from $v_s$ to $v_d$ is not the same for both graphs.
		Let there be two paths connecting the two vertices. In graph G, one path consists of 8 edges with weights of 1
		while the other path consists of one edge with a weight of 9. In graph G, the shortest path involves traversing
		the path with 8 edges for a cost of 8 instead of traversing the single edge with a cost of 9. In contrast in graph H,
		the 8 edge path has each edge weighted at 2 with a path cost of 16 and the single edge is weighted at 10. Therefore,
		the shortest path will be acheived by traversing the single edge. Thefore, the shortest path from $v_s$ to $v_d$ is
		different in G and H.