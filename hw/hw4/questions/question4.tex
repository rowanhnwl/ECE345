\section{Question 4}

In this question, we have three constraints for our flow network:
\\ 1. Each doctor should only be assigned to work on the days when they are available.
\\ 2. For a given parameter $c$, each doctor should be assigned to work at most $c$ vacation days.
\\ 3. For each vacation period $j$, any given doctor should be scheduled at most one day in $D_j$. 

First, we solve the problem without considering the third condition. Assume for doctor $i$, $A_i$ would represent
the set of vacation days that the doctor is available to work. To begin with, the source node is connected to the column
of doctor nodes ($d_i$ for doctor $i$). The edge for this connection is set to $c_i$ for each doctor. This way, we ensure that each doctor
would work less or equal to their capacity, covering constraint number two. Then, each doctor node is connected to the next layer which is the vacation
days ($v_{jm}$ for representing the $m$th day of $j$th vacation period), but only for the days that they are available (i.e. there is no connection if
the day is not part of $A_i$). The edge capacity for this connection is set to 1, representing one whole day in the vacation period.
The vacation days are then connected to the sink with an edge capacity of 1. A simplified version of this network flow is presented in Figure 1 with three doctors 
and two vacation periods. For solving the problem, we need to set a demanding value of $N$ (the total number of vacation days) for the sink's inflow. In other words, as we solve for the
maximum flow, if the inflow of the sink would be equal to $N$, we know that we have a feasible schedule with at least one doctor for every vacation day. If the sink value would be $k$ units less than $N$, it means
that there are $k$ vacation days that we do not have a doctor available for (i.e. there is no flow in $k$ vacation days). 

\vspace*{0.2cm}
{\centering \begin{tikzpicture}[
    mycircle/.style={
        circle,
        draw=black,
        fill=gray,
        fill opacity = 0.3,
        text opacity=1,
        inner sep=0pt,
        minimum size=40pt,
        font=\large},
    myarrow/.style={-Stealth},
    node distance=0.6cm and 1.2cm
    ]
    \node[mycircle] (s) {$s$};
    \node[mycircle,right=of s] (d2) {$d_2$};
    \node[mycircle,above=of d2] (d1) {$d_1$};
    \node[mycircle,below=of d2] (d3) {$d_3$};
    \node[mycircle,right=of d1] (v12) {$v_{12}$};
    \node[mycircle,above=of v12] (v11) {$v_{11}$};
    \node[mycircle,below=of v12] (v13) {$v_{13}$};
    \node[mycircle,right=of d3] (v21) {$v_{21}$};
    \node[mycircle,below=of v21] (v22) {$v_{22}$};
    \node[mycircle,right=of v13] (t) {$t$};

    \foreach \i/\j/\txt/\p in {% start node/end node/text/position
    s/d1/$c_1$/above,
    s/d2/$c_2$/above,
    s/d3/$c_3$/below,
    v11/t/1/above,
    v12/t/1/above,
    v13/t/1/above,
    v21/t/1/above,
    v22/t/1/above,
    d1/v11/1/above,
    d1/v12/1/above,
    d2/v12/1/below,
    d2/v22/1/below,
    d3/v13/1/above,
    d3/v21/1/above}
    \draw [myarrow] (\i) -- node[sloped,font=\large,\p] {\txt} (\j);

\end{tikzpicture} \par}
{\centering Figure 1: Flow network solution ignoring constraint number 3.\par}

Now that we have a valid graph for the easier version, it is time to tackle constraint number 3. For this part,
we have to ensure that each doctor works only on one day during a vacation period. To clarify this constraint using our previous example in Figure 1,
doctor $d_1$ has to work either on $v_{11}$ or $v_{12}$ regardless of how many days they can work during the vacations ($c_1$). For doing this, we need 
a new layer to limit the flow that goes into each vacation period from each doctor node. Thus, we introduce a layer between the doctors and the vacation days
to represent vacation period availability for each doctor ($p_{ij}$ is the node for $i$th doctor availibity for $j$th vacation period). The edges are again all set
to one to ensure the doctor works for only one day per vacation period (i.e. since the inflow of a vacation period is one, only one of the
vacation days would have a flow). The new version of our flow is represented in Figure 2, derived from our previous example. 
Again for comleting the algorithm, we set a demanding value of $N$ (the total number of vacation days) for the sink's inflow, knowing that our schedule is feasible as long as this
requirement (demand) is met in our circulation. As we use the Ford-Fulkerson polynomial-time algorithm for this question, it requires at most
$N$ iterations to find all the augmenting paths in the residual network. Thus, according to the textbook, the overall time complexity would be $\mathcal{O}(EN)$ ($E$ is the total number of edges) plus the required time
for checking if the sink's input flow is equal to $N$ or not.

\vspace*{0.2cm}
{\centering \begin{tikzpicture}[
    mycircle/.style={
        circle,
        draw=black,
        fill=gray,
        fill opacity = 0.3,
        text opacity=1,
        inner sep=0pt,
        minimum size=40pt,
        font=\large},
    myarrow/.style={-Stealth},
    node distance=0.6cm and 1.2cm
    ]
    \node[mycircle] (s) {$s$};
    \node[mycircle,right=of s] (d2) {$d_2$};
    \node[mycircle,above=of d2] (d1) {$d_1$};
    \node[mycircle,below=of d2] (d3) {$d_3$};
    \node[mycircle,above right=of d2] (p21) {$p_{21}$};
    \node[mycircle,below right=of d2] (p22) {$p_{22}$};
    \node[mycircle,above=of p21] (p12) {$p_{12}$};
    \node[mycircle,above=of p12] (p11) {$p_{11}$};
    \node[mycircle,below=of p22] (p31) {$p_{31}$};
    \node[mycircle,below=of p31] (p32) {$p_{32}$};

    \node[mycircle,right=of p12] (v12) {$v_{12}$};
    \node[mycircle,above=of v12] (v11) {$v_{11}$};
    \node[mycircle,below=of v12] (v13) {$v_{13}$};
    \node[mycircle,right=of p22] (v21) {$v_{21}$};
    \node[mycircle,below=of v21] (v22) {$v_{22}$};
    \node[mycircle,above right=of v21] (t) {$t$};

    \foreach \i/\j/\txt/\p in {% start node/end node/text/position
    s/d1/$c_1$/above,
    s/d2/$c_2$/above,
    s/d3/$c_3$/below,
    d1/p11//above,
    d1/p12//above,
    d2/p21//above,
    d2/p22//above,
    d3/p31//above,
    d3/p32//below,
    v11/t//above,
    v12/t//above,
    v13/t//above,
    v21/t//above,
    v22/t//above,
    p11/v11//above,
    p11/v12//above,
    p21/v12//below,
    p22/v22//above,
    p31/v13//above,
    p32/v21//above}
    \draw [myarrow] (\i) -- node[sloped,font=\large,\p] {\txt} (\j);

\end{tikzpicture} \par}
{\centering Figure 2: Flow network full solution, derived from Figure 1 (For more clarity, only the edges with weight not equal to 1 are specified).\par}