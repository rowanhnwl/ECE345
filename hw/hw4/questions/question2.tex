\section{Question 2}
The shortest path from every node in North America $v_{NA}$ to every node in Europe $v_E$, must include the edge $e_{transatlantic}$.
Additionally, both $v_{NA1}$ and $v_{E1}$ will be in the set of verticies for each shortest path since it is the only path connecting the two subgraphs $G_{NA}$ and $G_{E}$.
\linebreak
Using the above statements, the shortest path from each $v_{NA} \in G_{NA}$ to each $v_E \in G_E$ can be broken down into the following:
$shortest path from v_{NA} to v_{NA1} + shortest path from v_E to v_{E1} + e_{transatlantic}$.
\linebreak

The shortest path from $v_{NA} to v_{NA1}$ can be found in $\mathcal{O}((V + E)\lg{V})$ using Dijkstra's algorithm (CLRS 3rd Ed, pg 661 - 662) with $e_{transatlantic}$ removed.
Similarly, the shortest path from $v_{E} to v_{E1}$ can be found in $\mathcal{O}((V + E)\lg{V})$ using Dijkstra's algorithm with $e_{transatlantic}$ removed.
The shortest path can then be found by concatenating the previous two paths found with $e_{transatlantic}$.

\linebreak
Since two $\mathcal{O}((V + E)\lg{V})$ operations are run at most (in fact the amortized cost is less - not asymptotically - because 
both subgraphs only run Dijkstra's algorithm on their nodes and edges which is less the E and V), the asymptotic time complexity of the algorithm is $\mathcal{O}((V + E)\lg{V})$.
Dijkstra's algorithm uses a priority-queue of vertices to determine the next vertex with the smallest cost which takes $\mathcal{O}(V)$ space.