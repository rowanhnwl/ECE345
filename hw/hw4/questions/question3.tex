
\section{Question 3}
First, take each conversion rate and construct a directed graph using an adjacency list.
Each node represents a currency and its edges represents the conversion ratio from the currency of node A to node B.
For instance, inserting the conversion CAD to USD involves creating two edges with each edge specifying the conversion from and to both currencies.
An adjacency list is used because it keeps BFS and DFS down to a $\mathcal{O}(V + E)$ time complexity, 
it takes asymptotically less space and time to create than an adjacency matrix.
\linebreak
To resolve a currency exchange query, a DFS or BFS (both have take $\mathcal{O}(V + E)$ time) is run on the first node in the exchange query and
terminating if the second node is found in the search. If DFS/BFS fail to find the second currency, then there is no conversion possible.
If the search is successful, the currency exchange rate will be the muliplication of each edge weight.
\linebreak
The algorithm takes $\mathcal{O}(V + E)$ time because for adjacency lists, inserting an edge and a vertex takes constant time and V + E insertions
are done to construct the graph. The time complexity of the search is also $\mathcal{O}(V + E)$.
The algorithm takes $\mathcal{O}(V + E)$ space because only an adjacency list is used to store the currency exchange rates and their relationships.
